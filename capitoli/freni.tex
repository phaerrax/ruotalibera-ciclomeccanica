\chapter{Freni}
Quattro tipi di freno:
\begin{itemize}
  \item a bacchette, difficili da sostituire, presenti su vecchi modelli di bici,
  \item a tamburo, inseriti nel mozzo,
  \item a disco, più attuali, hanno bisogno di cerchioni appositi,
  \item ``classici'', i più diffusi sulle bici attuali.
\end{itemize}
Questi ultimi si distinguono per la struttura del meccanismo che muove le pastiglie, che esiste di tre tipi:
\begin{itemize}
  \item \textit{caliber}, classico sulle bici da città,
  \item \textit{cantilever}, più tipico sulle mountain bike,
  \item \textit{V-brake}, come il precedente.
\end{itemize}

Ci occuperemo perlopiù dei freni \textit{caliber}, più facili per quanto riguarda il montaggio e la manutenzione.

\section{Freni \textit{caliber}}
Il freno funziona con le leve sul manubrio che tendono dei cavi, collegati al ``blocco'' del freno, che a sua volta si muove e preme le pastiglie contro il cerchione, frenandone la rotazione.
I fili sono di un qualche metallo, protetti da una guaina, che allo stesso tempo fa da ``guida'' in cui il cavo scorre.

\subsection{Sostituire i fili}
Il cavo va cambiato quando:
\begin{itemize}
  \item si rompe,
  \item il freno non si chiude o non si riapre (per ruggine, sporco, o altro).
\end{itemize}
In caso di ruggine, ad esempio se la bici è rimasta molto tempo all'esterno, sono da cambiare sia filo che guaina; se è invece dello sporco a dare problemi pul essere sufficiente dare una bella pulita.

Innanzitutto diciamo che i fili sfilacciati sono molto difficili da inserire in una guaina, perciò sarebbero da cambiare.
Poi: la prima cosa da fare è chiaramente rimuovere il filo già presente.
Si sfila la testa dalla leva sul manubrio: in alcune bici, in particolare le mountain bike, per estrarre il filo bisogna allineare le fessure ruotando opportunamente le componenti della leva del freno, finché il filo può uscire.
A quel punto, tirando la leva, si può estrarre anche la testa del filo dal suo alloggiamento.
Essa si trova tipicamente in due forme, a ``barilotto'' (mountain bike) o a sfera (da città), e si aggancia alla leva per fissare il cavo.
Con un po' di movimento si estrae, dopodiché si sfila il resto del filo dal telaio.

Ora vediamo come montare il filo nuovo.
Innanzitutto ci sono due lunghezze per i cavi, poiché i freni anteriori, essendo più vicini al manubrio, necessitano di una minore lunghezza.
Prima di tutto si infila la testa del filo nella leva; alcune di esse, poi, vogliono un ``cappuccio'' che fissi la guaina nella leva: questo è necessario, altrimenti non si riuscirà a frenare.
Una volta inserita la testa del filo, infilarlo nella guaina.
Uno strato di grasso attorno al filo, dato prima di infilarlo, lo proteggerà dallo sporco e lo farà scorrere meglio.

La guaina va quindi prima di tutto tagliata della misura giusta: come determinarla?
La lunghezza del cavo del freno (e quindi della guaina) dovrà essere tale che esso non faccia curve troppo larghe o strette, pena una perdita di efficienza in frenata.
Raccomandiamo di misurare anche tenendo conto che sterzando la distanza tra il freno e la leva si modifica, quindi provare la lunghezza anche con il manubrio ruotato.

Se il filo (più che altro per i freni posteriori) è fatto passare all'interno del telaio, è bene non rimuovere mai tutte le sue componenti quando lo si smonta, ma lasciare sempre o il vecchio cavo o la vecchia guaina all'interno: questo ci permetterà in fase di montaggio di quello nuovo di avere una ``guida'' per far passare il filo all'interno.
Altrimenti sarà difficile far trovare al filo il foro di uscita dal telaio!
In tal caso, comunque ci si può aiutare con l'estremità uncinata di un raggio.

Al termine del filo possiamo inserire dei \emph{capiguaina}: sono dei cappucci che aiutano la guaina ad essere più robusta.
Non sempre si riesce a farli entrare al capo del filo del freno, in tal caso si possono anche omettere.
Talvolta già nel freno c'è un cappuccio in cui inserire il cavo del freno per fissarlo, e allora tale cappuccio farà le veci del capoguaina.

La guaina si può tranquillamente tagliare con un tronchese: basta essere ben decisi quando la si taglia, per fare un taglio netto, così da non sfilacciarla, altrimenti si rischia di schiacciare le pareti laterali occludendo così il buco in cui infilare il cavo.
In tal caso, usare un raggio per liberare il foro.

Una volta che la guaina è stata regolata e il filo infilato, si ultima il montaggio infilando il capo ancora scoperto del filo nel meccanismo che lo fissa al freno.
Esistono diversi tipi di fissaggio: uno abbastanza comune consiste in un dado, che, una volta avvitato, preme il cavo contro la struttura del freno e lo tiene fermo.
Quando tutto è stato fissato, si può tagliare il filo in eccesso, concludendo il montaggio.

\section{Pastiglie}
Quando si montano le pastigle, si deve notare che esse non sono dei parallelepipedi, ma un lato è inclinato in modo che esse si adattino bene al cerchione.
Come fissarle?
Si tira il cerchione finché la pastiglia è ben a contatto con esso: in questo modo essa sarà orientata nel modo corretto.
Dopodiché si tirano i capiguaina ``in posizione'' e si stringono le viti per fissarla.
Quando si lascia andare il cerchione, tornerà nella sua posizione originale, e la pastiglia tornerà ad essere un po' staccata, alla giusta distanza, da esso.
Se si tengono sempre la pastiglia e il resto ben fermi, si evita tra l'altro che, al momento di avvitare la prima, si storti il tutto.

Quando si chiude il dado centrale del freno, si deve stare attenti che le pastiglie siano ben centrate e simmetriche.
Il dado al centro dietro al freno si può stringere a piacere, tanto esso schiaccia il telaio e non il freno, quindi non modifica la mobilità delle pastiglie.
Quello al centro davanti, invece, regola appunto il grado di mobilità delle pastiglie, quindi non deve essere stretto fino in fondo: se è troppo stretto infatti il freno non si chiude e le pastiglie non toccano il cerchione, mentre se è troppo largo le levette si muovono ``male'', ``ballano'', quando si frena, compromettendo la frenata.
Quando si monta il freno, si stringe il secondo dado sul primo, che fa sì che il primo --- quello sotto --- non si sviti con l'uso. % Non so a quale passaggio si riferisca questa frase...

Con l'usura, le pastiglie si assottiglieranno, per cui il punto di contatto con il cerchione diventerà man mano più distante: si può in questo caso agire sul filetto per regolare finemente la tensione del filo del freno, avvicinando o allontanando di poco la guaina e quindi la pastiglia.
Alcune leve permettono di eseguire questa regolazione fine anche direttamente su di esse oltre che sul freno.
