\chapter{Ruote}
Il diametro riportato sul lato dello pneumatico indica su quali cerchioni è possibile montarlo.
Per quanto riguarda lo spessore si è più flessibili: su uno stesso cerchione è possibile montare pneumatici di differente diametro.

Controllare le camere d'aria significa controllare che siano giusti:
\begin{itemize}
  \item il diametro,
  \item la larghezza del tubo, che è indicata come un \emph{intervallo}, e tutti i copertoni entro la fascia indicata andranno bene con la camera d'aria.
\end{itemize}

\section*{Cerchioni}
\subsection*{Aprire i cerchioni}
Si comincia preferibilmente lontani dalla valvola, perché il copertone è più morbido.
Con un \emph{cacciacopertone} si crea una leva con cui spingere il copertone via dalla sua sede nel cerchione: una volta che esso è sollevato, il cacciacopertone si fissa in posizione agganciandolo ai raggi.
Tipicamente ne serviranno almeno due.
Una volta fissati più cacciacopertoni, bisogna far uscire dal cerchione la parte di copertone che sta in mezzo tra essi.
La posizione dei cacciacopertoni va regolata in base alla rigidità del copertone: più esso è rigido, più si farà fatica se i cacciacopertoni si trovano lontani.
Una volta sollevato il primo pezzo, il resto del copertone si può togliere a mano oppure con altri cacciacopertoni.
Se si è fatto un buon lavoro, una volta infilato e agganciato il terzo cacciacopertone quello che è in mezzo dovrebbe cadere.
Con il copertone completamente sfilato a metà del cerchione, si fa dunque uscire la valvola dalla sua sede e si sfila del tutto la camera d'aria; tutto ciò senza spostare il copertone, che può rimanere lì dov'è.

Tolta la camera d'aria si può osservare che il cerchione, nel sul lato esterno al cerchio, è sempre coperto da un nastro elastico (o anche da nastro isolante): esso serve a proteggere la camera d'aria dalle punte e dai bulloni dei raggi che potrebbero altrimenti bucarla molto molto facilmente.

Il battistrada, se le sua scanalature sono ``direzionate'', esse devono puntare nella direzione del moto in avanti della bicicletta.

I cerchioni anteriori e posteriori sono diversi.
Il mozzo anteriore è simmetrico, quello posteriore no, poiché deve ospitare la trasmissione.

\subsection*{Rimontare i copertoni}
Come rimontare il copertone: la procedura è in pratica l'inverso dello smontaggio visto prima.
Innanzitutto, se non è ancora infilato a metà sul cerchione, infilarlo.
Poi infilare la camera d'aria (sarà più facile se è leggermente già gonfiata) dentro al copertone, assicurandosi che non si attorcigli.
Si parte dalla valvola, che è la parte un po' più complicata, per poi man mano infilare tutta la camera d'aria.
Una volta inserita tutta, mettere a posto anche il copertone.
Meglio non usare cacciacopertoni qui, per evitare di ``pizzicare'' la camera d'aria..
Se il copertone fatica ad entrare, soprattutto nella parte finale, provare a ``massaggiarlo'' vicino al punto in cui lo si deve infilare.

\section*{Camere d'aria}
\subsection*{Pressione di gonfiaggio}
In assenza di altre indicazioni, una pressione di 3 bar va bene.
Solitamente sul lato degli pneumatici si trovano scritti i valori massimi e minimi della pressione consigliata.
È sicuro stare mezzo bar meno del massimo per una guida confortevole in città o comunque su superfici poco impegnative.
Se si usa la bici su sterrato è invece preferibile scendere a 1 o 1,5 bar meno del massimo, per evitare forature.
La ruota posteriore deve sorreggere un carico maggiore e per questo motivo è buona cosa gonfiarla un po' di più della anteriore.

\subsection*{Forature}
Se la foratura è avvenuta vicino alla valvola, diciamo subito che sarà molto difficile rimediare ad essa.
Negli altri casi, munirsi della toppa, un gessetto (facoltativo), e del mastice.
Per individuare la foratura, o le forature, della camera d'aria (dopo averla estratta dal copertone!) è utile immergerla in un secchio d'acqua e cercare da dove escono le bolle.
Segnarsi il punto preciso del foro, quindi spalmare abbondante mastice attorno ad esso: deve essere sufficiente ad attaccare la toppa senza lasciare scoperto alcun punto.
Aspettare una decina di minuti che il mastice raggiunga lo stato desiderato, quindi applicare la toppa.
Una volta che è stata fatta aderire bene alla camera d'aria, premere forte con le mani per qualche minuto finché la toppa non si stacca più.
L'uso di una morsa premette di ridurre di molto i tempi di ``premitura''.

Se si sbaglia ad applicare la toppa, la strada migliore è toglierla e provare daccapo con un'altra toppa.
È sempre meglio evitare di mettere toppe su toppe, per non creare gobbe nella camera d'aria.
Semmai è consigliabile aggiungere una toppa minuscola nel punto lasciato scoperto (con lo stesso procedimento).

\subsection*{Valvole}
Esistono tre o quattro tipi di valvole standard per le camere d'aria.
Non tutti i cerchioni sono adatti a tutti i tipi di valvole.
In caso di necessità si può usare, con giudizio, un trapano per allargare il buco e accomodare anche le valvole che non ci starebbero altrimenti.
Alcune valvole restano dritte sul cerchione grazie a una filettatura che le fissa a quest'ultimo; altre restano così grazie alla pressione della camera d'aria che le sorregge.

\section{Raggi}
Quando si rompe un raggio, la tensione non uniforme nel cerchione rende più probabile che se ne rompano altri, peggiorando così ulteriormente la stortatura della ruota, quindi si deve riparare il prima possibile.

Esistono diversi tipi di raggiature: la più comune è detta ``passo quattro'', per motivi che capiremo tra poco.
Un raggio è agganciato al mozzo centrale % è proprio il mozzo?
tramite una ``capocchia'', a uncino, che lo ferma.
Essa può infilarsi da sopra o da sotto il buco, guardando il cerchione dal lato: in una corretta raggiatura i raggi sono agganciati in alternanza, uno sopra e uno sotto, uno sopra e uno sotto, e così via.
Dall'aggangio centrale al cerchione, il raggio poi si attacca al cerchione incrociando altri raggi, passandoci sopra o sotto: lo schema con cui questo accade non è casuale, ma è regolare.

Preso un raggio qualsiasi, chiamiamo \emph{raggi consecutivi} i due raggi, alla sinistra e destra di esso, che sono montati dalla stessa parte del cerchione e che puntano nella stessa direzione.
Nella raggiatura a passo quattro ogni raggio ne incrocia altri tre, e (contandoli dal centro all'esterno) passa sopra, sopra e poi sotto, oppure al contrario a seconda se la capocchia è agganciata sopra o sotto.
Attenzione che il primo passaggio sopra o sotto è molto vicino all'aggancio del raggio!
I raggi consecutivi sono montati esattamente con lo stesso schema, e si trovano \emph{ogni quattro raggi} (da cui il nome ``passo quattro'').

Nella circonferenza esterna, i raggi sono fissati al cerchione attraverso dei \emph{nippoli}, che sono avvitati al metallo: si infila il raggio nel nippolo e si avvita quest'ultimo (nel senso contrario rispetto al solito).
Quando si cambiano i raggi, al momento di infilarli nei nippoli, è buona cosa sgonfiare la camera d'aria, per evitare il rischio di forarla.

I raggi hanno lunghezze e diametri diversi. Di diametro perlomeno tipicamente se ne usa uno soltanto.

\section{Centratura}
Quando un raggio si rompe, la tensione sul cerchione rimane distribuita in un modo non uniforme, perciò esso potrebbe deformarsi.
Inoltre se i raggi sono tesi in modo diverso c'è il rischio che la ruota si ``ovalizzi''.

Con l'apposito attrezzo si può controllare il punto in cui la ruota non è ben centrata.
Si appoggia la ruota sui perni e si regolano le due viti in modo che esse segnino la posizione corretta in cui la ruota deve essere: poi si fa girare la ruota, e se essa tocca le viti allora nel punti di contatto è deformata e da sistemare: essa dovrà essere tirata nella direzione opposta a dove tocca, e per fare ciò si deve tendere il raggio della parte opposta, avvitando (con un \emph{tira raggi}) il relativo nippolo.
Normalmente mezzo giro del nippolo può essere sufficiente per legere deviazioni, quindi bisogna andare piano e controllare spesso che la ruota si sia ricentrata.
Siccome non si può tendere un raggio più di tanto, può essere utile agire su più raggi (vicini) contemporaneamente, oppure anziché tendere un raggio, ``smollare'' quello opposto (comunque, è sempre preferibile tendere).

Normalmente la ruota anteriore è simmetrica, ma quella posteriore no, poiché deve ospitare il sistema di trasmissione che è montato chiaramente solo da una parte, quindi fare attenzione a come si regolano le viti regolatrici.
La ruota posteriore potrebbe inoltre essere \emph{campanata}, ossia i raggi potrebbero tutti essere inclinati dalla parte opposta alle corone.
In questo caso è più complicato regolarla.
