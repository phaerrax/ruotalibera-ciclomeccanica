\chapter{Trasmissione}
La trasmissione comprente la corona, la catena e il pignone.
Esistono grosso modo due tipi di pignone, quello singolo e quello multiplo (quando la bicicletta ha il cambio posteriore).
Esso fa in modo che la ruota giri solo in un senso.

Anche di corone ce n'è una singola e una multipla, a seconda che ci sia o no il cambio anteriore.

Esistono più tipi di rapporti.

Rapporto anteriore: più grande è la corona su cui è posizionata la catena, maggiore è la durezza della pedalata.
Per il rapporto posteriore vale il contrario.

\section{Catena}
La catena per pignoni singoli o multipli è differente: tipicamente quella per i pignoni multipli è più sottile.

Il passo della catena, invece, è standard.

Controllo della catena: il passo deve essere regolare, e deve essere ben tesa; per verificare, si può provare a pedalare con la bici capovolta e provare a farla uscire dalla sua sede: va bene se non esce.
Se è troppo tesa va sistemata, c'è il rischio che si spezzi altrimenti.

La catena e il pignone si usurano con tempi diversi, quindi una catena nuova su un pignone vecchio può comunque saltare giù!

Controllare anche la \emph{linea catena}, ossia l'allineamento tra corona anteriore e posteriore.

\section{Cambio}
La catena è spostata lungo il pignone dal \emph{deragliatore}, che si muove agendo sulle leve del cambio.
Essa può avere degli scatti, che indicano il rapporto corrente, oppure può scorrere liscia dal massimo al minimo.

La guaina che ricopre il filo del cambio è solitamente più piccola di quella che si trova sui fili dei freni, ma in caso di necessità sono intercambiabili.

L'attacco del filo è invece come quello del freno.

Il filo deve essere teso ``come una corda di chitarra'' quando si trova in posizione neutra, ossia sul pignone più piccolo.
Quando il cavo è teso bene, agendo sulla leva del cambio la catena deve saltare sul pignone più alto; se no, il cavo va teso maggiormente.

Quando la tensione è corretta, si può passare a regolare le viti di fine corsa.
Esse possono avere una dicitura ``H'' o ``L'' che indica se si riferiscono al rapporto più duro o più molle.
Avvitando la vite ``H'' si blocca la catena nella direzione verso la bici; avvitando la ``L'' la si blocca nella direzione all'infuori.
Tipicamente si parte a regolare la ``H''.

Se la catena non sale quando si cambia bisogna regolare la tensione dei cavi del cambio; se invece la catena cade, bisogna regolare il fine corsa.

Può aiutare sgrassare le molle del cambio.
