\chapter{M-check}
L'\textit{M-check} è una serie di controlli generici da effettuare per controllare se la bicicletta è in buono stato.
Il nome deriva dal fatto che si segue un percorso a ``M'' guardando la bici dal lato: si controllano
\begin{enumerate}
  \item la ruota anteriore,
  \item il manubrio,
  \item i pedali,
  \item la sella,
  \item la ruota posteriore.
\end{enumerate}

\subsection*{Ruota anteriore}
La ruota non deve sfregare quando gira.
Non devono esserci moti nella direzione dell'asse della ruota: sono fonti di attrito e peggiorano l'efficienza della pedalata.

Controllare le pastiglie dei freni (per freni a pastiglia): devono ancora vedersi le scanalature.

\subsection*{Manubrio}
Lo sterzo deve essere duro ``al punto giusto'', grazie all'azione dei cuscinetti che si trovano all'interno del manubrio, che altrimenti andranno regolati.
Lo sterzo non deve inoltre avere dei giochi, ad esempio non deve essere ``smollato'' inclinandosi in avanti quando lo si spinge.
Insomma deve limitarsi a girare, senza muoversi in altre direzioni.

\subsection*{Pedali}
Anche i pedali, come per le ruote, non devono presentare moti nella direzione ortogonale alla lunghezza della bici: essi potrebbero causare, sollecitando la catena, la caduta di quest'ultima, e in ogni caso con questi movimenti sbagliati aumenta l'usura dei pezzi.

Le pedivelle devono inoltre essere ben connesse con il loro perno, e anch'esse non devono avere movimenti insoliti.

\subsection*{Sella}
L'altezza della sella dovrebbe essere regolata in modo che mentre si pedala, quando il piede si trova sul pedale più in basso, la gamba sia \emph{quasi} tesa, in modo da massimizzare l'efficienza della pedalata. Non troppo tesa, però, per non causare problemi di ipertensione ai muscoli.
Stare inoltre attenti, alzando la sella, a non superare (se c'è) la tacca con la dicitura ``MAX'' o simile, altrimenti il tubo che regge la sella potrebbe uscire mentre si va in giro, o comunque la sella potrebbe fare movimenti strani poiché non è ben fissata.
