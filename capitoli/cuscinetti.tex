\chapter{Cuscinetti}
I cuscinetti sono delle sferette che permettono alle parti rotanti della bicicletta di ruotare attorno a un asse con il minimo attrito: i pedali, il mozzo delle ruote e il manubrio.

Tali sferette si trovano collocate in una ``pista'' in cui sono vincolate a scorrere, oppure ingabbiate.

Le sfere esistono in varie dimensioni e tipicamente il movimento centrale ha quelle di dimensione più grande.

Le sfere devono sempre essere ben ingrassate in modo da non fare attrito.

\section{Mozzo}
Il mozzo è il corpo centrale delle due ruote, che gira sopra all'asse che invece non ruota, ma è fissato sul telaio.
Tramite i raggi, il mozzo trasmette il movimento alla ruota.

Esistono diversi tipi di mozzi (con cambio, a ruota libera, a scatto fisso, \emph{flip-flop}\ldots).

Sul mozzo, un dado particolare chiamato ``cono'' tiene in posizione nella pista le sfere.
Proseguendo lungo l'asse verso l'esterno, troviamo un distanziatore, e un controtado che tiene tutto ciò che sta all'interno in posizione.
Il ``pacco pignoni'' ruota appoggiandosi sul distanziatore.

Il controdado si avvita tenendo bloccato fermo il cono usando un'altra chiave, e va fissato solo quando si ha già messo il cono nella posizione giusta, ossia quando si vede che il mozzo ``ruota bene''.

L'asse può essere bucato nelle ruote a sgancio rapido.

I vari coni, dadi e controdadi devono essere ben fissati in modo che la ruota \emph{ruoti} (se sono troppo stretti si blocca e non gira sull'asse) ma che lo faccia senza giochi.

\section{Movimento centrale}
Le piste dei cuscinetti si trovano nelle \emph{calotte}.
Prima si posiziona la calotta destra (come per il mozzo posteriore); se non è possibile la si regola agendo dal lato sinistro.

La calotta ha un controdado chiamato \emph{ghiera}, che la tiene ferma.

Si avvita la calotta finché batte sul cuscinetto; l'altra pista per i cuscinetti (oltre alla calotta) sta direttamente sull'asse.

Alcuni movimenti centrali hanno tutto questo racchiuso in una cartuccia: se si rompe, va cambiata tutta assieme.

Esistono diversi tipi di assi: a sezione quadrata o circolare, che è anche la sezione dei perni delle pedivelle.

Sezione circolare: c'è una scanalatura dove si inserisce la pedivella; c'è poi un foro sulla pedivella in cui inserire una ``chiavella'' che si avvita e tiene tutto fermo sull'asse (potrebbe esserci bisogno di martellare un po').
Il filetto sulla chiavella serve per avvitare un dado dalla parte opposta, e fissarla lì dov'è.
